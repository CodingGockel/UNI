% ----------------------- TODO ---------------------------
% Diese Daten müssen pro Blatt angepasst werden:
\newcommand{\NUMBER}{1}
\newcommand{\EXERCISES}{4}
% Diese Daten müssen einmalig pro Vorlesung angepasst werden:
\newcommand{\COURSE}{Diskrete Strukturen I}
\newcommand{\TUTOR}{Jörg Bader}
\newcommand{\STUDENTA}{Moritz Hahn}
\newcommand{\STUDENTB}{}
\newcommand{\STUDENTC}{}
\newcommand{\DEADLINE}{19.10.2024}
% ----------------------- TODO ---------------------------

\documentclass[a4paper]{scrartcl}

\usepackage[utf8]{inputenc}
\usepackage[ngerman]{babel}
\usepackage{amsmath}
\usepackage{amssymb}
\usepackage{fancyhdr}
\usepackage{color}
\usepackage{graphicx}
\usepackage{lastpage}
\usepackage{listings}
\usepackage{tikz}
\usepackage{pdflscape}
\usepackage{subfigure}
\usepackage{float}
\usepackage{polynom}
\usepackage{hyperref}
\usepackage{tabularx}
\usepackage{forloop}
\usepackage{geometry}
\usepackage{listings}
\usepackage{fancybox}
\usepackage{tikz}
\usepackage{amsmath}
\usepackage{amsfonts}
\usepackage{amssymb}
\usepackage{float}
\restylefloat{table}
%Größe der Ränder setzen
\geometry{a4paper,left=3cm, right=3cm, top=3cm, bottom=3cm}

%Kopf- und Fußzeile
\pagestyle {fancy}
\fancyhead[L]{Tutor: \TUTOR}
\fancyhead[C]{\COURSE}
\fancyhead[R]{\today}

\fancyfoot[L]{}
\fancyfoot[C]{} 
\fancyfoot[R]{Seite \thepage /\pageref*{LastPage}}

%Formatierung der Überschrift, hier nichts ändern
\def\header#1#2{
  \begin{center}
    {\Large Übungsblatt #1}\\
    {(Abgabetermin #2)}
  \end{center}
}

%Definition der Punktetabelle, hier nichts ändern
\newcounter{punktelistectr}
\newcounter{punkte}
\newcommand{\punkteliste}[2]{%
  \setcounter{punkte}{#2}%
  \addtocounter{punkte}{-#1}%
  \stepcounter{punkte}%<-- also punkte = m-n+1 = Anzahl Spalten[1]
  \begin{center}%
  \begin{tabularx}{\linewidth}[]{@{}*{\thepunkte}{>{\centering\arraybackslash} X|}@{}>{\centering\arraybackslash}X}
      \forloop{punktelistectr}{#1}{\value{punktelistectr} < #2 } %
      {%
        \thepunktelistectr &
      }
      #2 &  $\Sigma$ \\
      \hline
      \forloop{punktelistectr}{#1}{\value{punktelistectr} < #2 } %
      {%
        &
      } &\\
      \forloop{punktelistectr}{#1}{\value{punktelistectr} < #2 } %
      {%
        &
      } &\\
    \end{tabularx}
  \end{center}
}

\begin{document}

\begin{tabularx}{\linewidth}{m{0.2 \linewidth}X}
  \begin{minipage}{\linewidth}
    \STUDENTA\\
    \STUDENTB\\
    \STUDENTC
  \end{minipage} & \begin{minipage}{\linewidth}
    \punkteliste{1}{\EXERCISES}
  \end{minipage}\\
\end{tabularx}

\header{Nr. \NUMBER}{\DEADLINE}



% ----------------------- TODO ---------------------------
% Hier werden die Aufgaben/Lösungen eingetragen:

\section*{Aufgabe 2}
\subsection*{(a)}

\paragraph{Binärdarstellung von 93:}
\[
93 \div 2 = 46 \ \text{Rest} \ 1 \quad \text{(niederwertigstes Bit)}
\]
\[
46 \div 2 = 23 \ \text{Rest} \ 0
\]
\[
23 \div 2 = 11 \ \text{Rest} \ 1
\]
\[
11 \div 2 = 5 \ \text{Rest} \ 1
\]
\[
5 \div 2 = 2 \ \text{Rest} \ 1
\]
\[
2 \div 2 = 1 \ \text{Rest} \ 0
\]
\[
1 \div 2 = 0 \ \text{Rest} \ 1
\]
Wenn man die Reste von unten nach oben liest, erhält man:
\[
93_{10} = 1011101_2
\]

\paragraph{Binärdarstellung von 162:}
\[
162 \div 2 = 81 \ \text{Rest} \ 0
\]
\[
81 \div 2 = 40 \ \text{Rest} \ 1
\]
\[
40 \div 2 = 20 \ \text{Rest} \ 0
\]
\[
20 \div 2 = 10 \ \text{Rest} \ 0
\]
\[
10 \div 2 = 5 \ \text{Rest} \ 0
\]
\[
5 \div 2 = 2 \ \text{Rest} \ 1
\]
\[
2 \div 2 = 1 \ \text{Rest} \ 0
\]
\[
1 \div 2 = 0 \ \text{Rest} \ 1
\]
Die Binärdarstellung von 162 ist also:
\[
162_{10} = 10100010_2
\]

\subsection*{(b)}

\paragraph{Hexadezimaldarstellung von 93 (\(1011101_2\)):}
\begin{itemize}
    \item Zunächst ergänzen wir auf volle 4er-Blöcke, indem wir vorne Nullen hinzufügen: \( 0101 \ 1101 \)
    \item \( 0101_2 = 5_{16} \)
    \item \( 1101_2 = D_{16} \)
\end{itemize}
Damit ist:
\[
93_{10} = 5D_{16}
\]

\paragraph{Hexadezimaldarstellung von 162 (\(10100010_2\)):}
\begin{itemize}
    \item Die 4er-Blöcke sind bereits vollständig: \( 1010 \ 0010 \)
    \item \( 1010_2 = A_{16} \)
    \item \( 0010_2 = 2_{16} \)
\end{itemize}
Damit ist:
\[
162_{10} = A2_{16}
\]

\subsection*{(c)}

\paragraph{Allgemeine Beobachtung:}
Wenn \( a + b = 2^n - 1 \), dann kann man feststellen, dass die Binärdarstellungen von \( a \) und \( b \) so beschaffen sind, dass sie sich in keinem Bit "überschneiden", also in keinem Bit beide eine \( 1 \) haben. In der Summe ergibt sich dann für jedes Bit \( a_i \) und \( b_i \), dass \( a_i + b_i = 1 \). Das bedeutet, dass die Binärdarstellungen von \( a \) und \( b \) genau komplementär zueinander sind, also dort, wo \( a \) eine \( 1 \) hat, hat \( b \) eine \( 0 \) und umgekehrt.

\paragraph{Beispiel mit \( n = 8 \), \( a = 93 \) und \( b = 162 \):}
\[
a = 1011101_2
\]
\[
b = 10100010_2
\]
Die Summe \( a + b \) ist:
\[
1011101_2 + 10100010_2 = 11111111_2 = 255_{10} = 2^8 - 1
\]
Es gilt also, dass \( a \) und \( b \) in ihren Binärdarstellungen komplementär zueinander sind.


\section*{Aufgabe 2}

\subsection*{(a)}

\begin{table}[H]
  \centering
  \begin{tabular}{llllll}
  / & 1 & 0 & 0 & 1 & 1 \\
  + & 0 & 1 & 1 & 1 & 0 \\
  1 & 0 & 0 & 0 & 0 & 1
  \end{tabular}
  \end{table}

Das Ergebnis der Addition im Binärsystem ist also:
\[
19_2 + 14_2 = 100001_2
\]
Im Dezimalsystem ergibt dies:
\[
100001_2 = 33_{10}
\]

\subsection*{(b) Schriftliche Multiplikation von 19 und 14}

\paragraph{Dezimalsystem:}
\[
19_{10} \times 14_{10} = 266_{10}
\]

\paragraph{Binärsystem:}
\[
19_{10} = [10011]_2
\]
\[
14_{10} = [1110]_2 = [01110]_2
\]
Nun führen wir die schriftliche Multiplikation im Binärsystem durch:

\[
\begin{array}{r}
    10011_2 \\
  \times \ 01110_2 \\
  \hline
     00000 \quad \text{(10011 mal 0)} \\
    10011  \quad \text{(10011 mal 1, um 1 Stelle nach links verschoben)} \\
   10011   \quad \text{(10011 mal 1, um 2 Stellen nach links verschoben)} \\
  10011    \quad \text{(10011 mal 1, um 3 Stellen nach links verschoben)} \\
  \hline
 100001110 \quad \text{(Ergebnis der Addition der Teilergebnisse)} \\
\end{array}
\]

Das Ergebnis der Multiplikation im Binärsystem ist also:
\[
10011_2 \times 1110_2 = 100001110_2
\]
Im Dezimalsystem ergibt dies:
\[
100001110_2 = 266_{10}
\]

\section*{Aufgabe 3}

\subsection*{(a)}

Die Binärdarstellung lautet $[1011wxyz]_2$, wobei $w$, $x$, $y$ und $z$ beliebige Binärziffern (0 oder 1) sind. Das bedeutet, dass die ersten vier Stellen festgelegt sind: $1011$ (was 11 im Dezimalsystem entspricht) und die letzten vier Stellen beliebig sein können. Damit gibt es $2^4 = 16$ mögliche Kombinationen für $wxyz$.

Für jede dieser Kombinationen können wir die Zahl berechnen:

\[
[1011wxyz]_2 = 11 \times 16 + (wxyz)_2
\]

Damit liegen die Dezimalwerte zwischen:

\[
11 \times 16 + 0 = 176 \quad \text{und} \quad 11 \times 16 + 15 = 191
\]

Also sind alle Dezimalzahlen zwischen 176 und 191.

\subsection*{(b)}

Hier sind die vier Ziffern $xy$ beliebig. Die festgelegten Ziffern sind $101000$, was 40 im Dezimalsystem entspricht. Die ersten beiden Stellen $xy$ können wieder beliebig sein, was $2^2 = 4$ mögliche Kombinationen ergibt.

Für jede dieser Kombinationen können wir die Zahl berechnen:

\[
[xy101000]_2 = (xy)_2 \times 64 + 40
\]

Die möglichen Werte für $(xy)_2$ sind $00, 01, 10, 11$, also:

\[
0 \times 64 + 40 = 40
\]
\[
1 \times 64 + 40 = 104
\]
\[
2 \times 64 + 40 = 168
\]
\[
3 \times 64 + 40 = 232
\]

Also sind die Dezimalzahlen 40, 104, 168 und 232.

\subsection*{(c)}

Wenn in der Binärdarstellung kein $01$ vorkommen darf, dann müssen die Zahlen ausschließlich aus Folgen von Einsen oder Nullen bestehen. Die zulässigen Muster sind also $00000000, 11111111, 11111110, 11111100, \dots$ und umgekehrt mit Nullen.

Die Zahlen, die dieses Muster erfüllen, sind:
\[
0, 1, 3, 7, 15, 31, 63, 127, 255
\]

\subsection*{(d)}

Die Zahl $[11110111]_2$ entspricht $247_{10}$. Nun suchen wir Zahlen, die sich in genau einer Stelle von dieser Binärzahl unterscheiden. Das bedeutet, dass wir eine der 8 Stellen von $[11110111]_2$ umdrehen können, um die gesuchten Zahlen zu erhalten.

\[
11110111 \quad (247)
\]
\[
01110111 = 119, \quad 10110111 = 183, \quad 11010111 = 215
\]
\[
11100111 = 231, \quad 11111111 = 255, \quad 11110011 = 243, \quad 11110101 = 245
\]

Die gesuchten Zahlen sind also: $119, 183, 215, 231, 255, 243, 245$.

\subsection*{(e)}

Die Binärdarstellung $[v10wx1yz]_2$ hat 5 feste Bits ($10$ und $1$ sind festgelegt), und die restlichen Bits $v, w, x, y, z$ sind frei wählbar. Das ergibt $2^5 = 32$ mögliche Kombinationen.

Also gibt es 32 verschiedene Dezimalzahlen.

\subsection*{(f)}

Die Zahl $[11110111]_2$ entspricht $247_{10}$. Um zwei Stellen zu ändern, wählen wir zwei der 8 möglichen Stellen aus, die wir umdrehen können. Die Anzahl der möglichen Kombinationen, zwei Stellen auszuwählen, ist:

\[
\binom{8}{2} = \frac{8 \times 7}{2} = 28
\]

Also gibt es 28 verschiedene Dezimalzahlen.

\section*{4}

\subsection*{(a) }

Eine Subnetzmaske besteht aus einer Folge von Einsen gefolgt von Nullen in der binären Darstellung. Entscheiden wir, welche der beiden Zahlenfolgen erlaubte Subnetzmasken sind:

\begin{itemize}
    \item \textbf{i.} $255.255.200.0$: \\
    Die dezimale Darstellung lautet:
    \[
    255 = 11111111_2, \quad 255 = 11111111_2, \quad 200 = 11001000_2, \quad 0 = 00000000_2
    \]
    Diese Zahlenfolge ist keine gültige Subnetzmaske, da die Einsen in der binären Darstellung von $200$ nicht fortlaufend sind (es gibt eine Null zwischen den Einsen).
    
    \item \textbf{ii.} $252.0.0.0$: \\
    Die dezimale Darstellung lautet:
    \[
    252 = 11111100_2, \quad 0 = 00000000_2, \quad 0 = 00000000_2, \quad 0 = 00000000_2
    \]
    Diese Zahlenfolge ist eine gültige Subnetzmaske, da die Einsen fortlaufend sind, gefolgt von Nullen.
\end{itemize}

\subsection*{(b) }

Die dezimalen Zahlen in einer IPv4-Subnetzmaske sind die jeweiligen Werte der 8-Bit-Blöcke in der binären Darstellung. Da jeder Block 8 Bits lang ist, können die dezimalen Werte zwischen 0 und 255 liegen. Diese Zahlen müssen jedoch fortlaufende Einsen und dann Nullen darstellen, was bedeutet, dass nur bestimmte Werte zulässig sind. Diese Werte sind:

\[
0, 128, 192, 224, 240, 248, 252, 254, 255
\]

\subsection*{(c)}

Die Subnetzmaske $255.255.224.0$ hat folgende binäre Darstellung:

\[
255 = 11111111_2, \quad 255 = 11111111_2, \quad 224 = 11100000_2, \quad 0 = 00000000_2
\]
Also lautet die vollständige binäre Darstellung:

\[
11111111.11111111.11100000.00000000
\]

Nun überprüfen wir, ob sich die beiden IP-Adressen jeweils im selben Subnetz befinden können, indem wir die IP-Adressen mit der Netzmaske logisch UND verknüpfen.

\subsubsection*{i. IP-Adressen: 10.10.138.10 und 10.10.158.248}

Die binäre Darstellung der beiden IP-Adressen lautet:

\[
10.10.138.10 = 00001010.00001010.10001010.00001010
\]
\[
10.10.158.248 = 00001010.00001010.10011110.11111000
\]

Nun wenden wir die Subnetzmaske an:

\[
00001010.00001010.10001010.00001010 \land 11111111.11111111.11100000.00000000 = \] \\
\[
00001010.00001010.10000000.00000000
\]
\[
00001010.00001010.10011110.11111000 \land 11111111.11111111.11100000.00000000 = \] \\
\[
00001010.00001010.10000000.00000000
\]

Da beide IP-Adressen nach der Anwendung der Subnetzmaske denselben Wert haben, befinden sie sich im selben Subnetz.

\subsubsection*{ii. IP-Adressen: 120.210.68.0 und 120.210.58.0}

Die binäre Darstellung der beiden IP-Adressen lautet:

\[
120.210.68.0 = 01111000.11010010.01000100.00000000
\]
\[
120.210.58.0 = 01111000.11010010.00111010.00000000
\]

Nun wenden wir die Subnetzmaske an:

\[
01111000.11010010.01000100.00000000 \land 11111111.11111111.11100000.00000000 = \] \\
\[
01111000.11010010.01000000.00000000
\]
\[
01111000.11010010.00111010.00000000 \land 11111111.11111111.11100000.00000000 = \] \\
\[
01111000.11010010.00100000.00000000
\]

Da die beiden IP-Adressen nach der Anwendung der Subnetzmaske unterschiedliche Werte haben, befinden sie sich nicht im selben Subnetz.

\subsection*{(d)}

Eine IPv4-Adresse hat 32 Bit, was bedeutet, dass es insgesamt $2^{32}$ mögliche IPv4-Adressen gibt:

\[
2^{32} = 4,294,967,296
\]

also ist es nicht möglich, jedem Smartphone eine eindeutige IPv4-Adresse zuzuweisen.

Eine IPv6-Adresse besteht aus 128 Bit. Die Anzahl der möglichen IPv6-Adressen ist daher:

\[
2^{128} = 3.4 \times 10^{38}
\]

Wenn wir diese Anzahl durch 7 Milliarden teilen, erhalten wir die Anzahl der IPv6-Adressen pro Smartphone:

\[
\frac{2^{128}}{7 \times 10^9} \approx 4.9 \times 10^{28}
\]

Also könnte theoretisch jedes Smartphone etwa $4.9 \times 10^{28}$ IPv6-Adressen erhalten.
\end{document}
%%% Local Variables:
%%% mode: latex
%%% TeX-master: t
%%% End:
