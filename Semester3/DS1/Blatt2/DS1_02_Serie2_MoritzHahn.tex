% ----------------------- TODO ---------------------------
% Diese Daten müssen pro Blatt angepasst werden:
\newcommand{\NUMBER}{2}
\newcommand{\EXERCISES}{4}
% Diese Daten müssen einmalig pro Vorlesung angepasst werden:
\newcommand{\COURSE}{Diskrete Strukturen I}
\newcommand{\TUTOR}{Jörg Bader}
\newcommand{\STUDENTA}{Moritz Hahn}
\newcommand{\STUDENTB}{}
\newcommand{\STUDENTC}{}
\newcommand{\DEADLINE}{26.10.2024}
% ----------------------- TODO ---------------------------

\documentclass[a4paper]{scrartcl}

\usepackage[utf8]{inputenc}
\usepackage[ngerman]{babel}
\usepackage{amsmath}
\usepackage{amssymb}
\usepackage{fancyhdr}
\usepackage{color}
\usepackage{graphicx}
\usepackage{lastpage}
\usepackage{listings}
\usepackage{tikz}
\usepackage{pdflscape}
\usepackage{subfigure}
\usepackage{float}
\usepackage{polynom}
\usepackage{hyperref}
\usepackage{tabularx}
\usepackage{forloop}
\usepackage{geometry}
\usepackage{listings}
\usepackage{fancybox}
\usepackage{tikz}
\usepackage{amsmath}
\usepackage{amsfonts}
\usepackage{amssymb}
\usepackage{float}
\restylefloat{table}
%Größe der Ränder setzen
\geometry{a4paper,left=3cm, right=3cm, top=3cm, bottom=3cm}

%Kopf- und Fußzeile
\pagestyle {fancy}
\fancyhead[L]{Tutor: \TUTOR}
\fancyhead[C]{\COURSE}
\fancyhead[R]{\today}

\fancyfoot[L]{}
\fancyfoot[C]{} 
\fancyfoot[R]{Seite \thepage /\pageref*{LastPage}}

%Formatierung der Überschrift, hier nichts ändern
\def\header#1#2{
  \begin{center}
    {\Large Übungsblatt #1}\\
    {(Abgabetermin #2)}
  \end{center}
}

%Definition der Punktetabelle, hier nichts ändern
\newcounter{punktelistectr}
\newcounter{punkte}
\newcommand{\punkteliste}[2]{%
  \setcounter{punkte}{#2}%
  \addtocounter{punkte}{-#1}%
  \stepcounter{punkte}%<-- also punkte = m-n+1 = Anzahl Spalten[1]
  \begin{center}%
  \begin{tabularx}{\linewidth}[]{@{}*{\thepunkte}{>{\centering\arraybackslash} X|}@{}>{\centering\arraybackslash}X}
      \forloop{punktelistectr}{#1}{\value{punktelistectr} < #2 } %
      {%
        \thepunktelistectr &
      }
      #2 &  $\Sigma$ \\
      \hline
      \forloop{punktelistectr}{#1}{\value{punktelistectr} < #2 } %
      {%
        &
      } &\\
      \forloop{punktelistectr}{#1}{\value{punktelistectr} < #2 } %
      {%
        &
      } &\\
    \end{tabularx}
  \end{center}
}

\begin{document}

\begin{tabularx}{\linewidth}{m{0.2 \linewidth}X}
  \begin{minipage}{\linewidth}
    \STUDENTA\\
    \STUDENTB\\
    \STUDENTC
  \end{minipage} & \begin{minipage}{\linewidth}
    \punkteliste{1}{\EXERCISES}
  \end{minipage}\\
\end{tabularx}

\header{Nr. \NUMBER}{\DEADLINE}

Die vier zusammengesetzten Aussagen lauten:
\section*{Aufgabe 1}

\subsection*{(a)}
\[
(A \land B) \rightarrow C
\]
\textbf{Kontraposition:} $\neg C \rightarrow \neg (A \land B)$ \\
\textbf{In natürlicher Sprache:} Wenn der Benutzer nicht auf die Datenbank zugreifen kann, dann ist er entweder nicht angemeldet oder hat nicht die erforderlichen Rechte.

\subsection*{(b)}
\[
\neg A \lor B
\]
(Keine Implikation, daher keine Kontraposition erforderlich.)

\subsection*{(c)}
\[
\neg A \rightarrow \neg C
\]
\textbf{Kontraposition:} $C \rightarrow A$ \\
\textbf{In natürlicher Sprache:} Wenn der Benutzer auf die Datenbank zugreifen kann, dann ist er angemeldet.

\subsection*{(d)}
\[
C \rightarrow (A \land B)
\]
\textbf{Kontraposition:} $\neg (A \land B) \rightarrow \neg C$ \\
\textbf{In natürlicher Sprache:} Wenn der Benutzer nicht angemeldet ist oder nicht die erforderlichen Rechte hat, kann er nicht auf die Datenbank zugreifen.


\section*{Aufgabe 2}
\subsection*{(a)}

\begin{tabular}{|c|c|c|c|c|}
\hline
$A$ & $B$ & $A \land B$ & $\neg (A \land B)$ & $\neg A \lor \neg B$ \\
\hline
W & W & W & F & F \\
W & F & F & W & W \\
F & W & F & W & W \\
F & F & F & W & W \\
\hline
\end{tabular}

\textbf{Ergebnis}: $\neg (A \land B)$ und $\neg A \lor \neg B$ haben in allen Fällen die gleichen Werte, also gilt $\neg (A \land B) \Leftrightarrow \neg A \lor \neg B$.

\subsection*{(b)}

\begin{tabular}{|c|c|c|c|c|}
\hline
$A$ & $B$ & $A \lor B$ & $\neg (A \lor B)$ & $\neg A \land \neg B$ \\
\hline
W & W & W & F & F \\
W & F & W & F & F \\
F & W & W & F & F \\
F & F & F & W & W \\
\hline
\end{tabular}

\textbf{Ergebnis}: $\neg (A \lor B)$ und $\neg A \land \neg B$ haben in allen Fällen die gleichen Werte, also gilt $\neg (A \lor B) \Leftrightarrow \neg A \land \neg B$.

\subsection*{(c)}

\begin{tabular}{|c|c|c|c|c|c|c|c|}
\hline
$A$ & $B$ & $C$ & $B \land C$ & $A \lor (B \land C)$ & $A \lor B$ & $A \lor C$ & $(A \lor B) \land (A \lor C)$ \\
\hline
W & W & W & W & W & W & W & W \\
W & W & F & F & W & W & W & W \\
W & F & W & F & W & W & W & W \\
W & F & F & F & W & W & W & W \\
F & W & W & W & W & W & W & W \\
F & W & F & F & F & W & F & F \\
F & F & W & F & F & F & W & F \\
F & F & F & F & F & F & F & F \\
\hline
\end{tabular}

\textbf{Ergebnis}: $A \lor (B \land C)$ und $(A \lor B) \land (A \lor C)$ haben in allen Fällen die gleichen Werte, also gilt $A \lor (B \land C) \Leftrightarrow (A \lor B) \land (A \lor C)$.

\subsection*{(d)}

\begin{tabular}{|c|c|c|c|c|c|c|c|}
\hline
$A$ & $B$ & $C$ & $B \lor C$ & $A \land (B \lor C)$ & $A \land B$ & $A \land C$ & $(A \land B) \lor (A \land C)$ \\
\hline
W & W & W & W & W & W & W & W \\
W & W & F & W & W & W & F & W \\
W & F & W & W & W & F & W & W \\
W & F & F & F & F & F & F & F \\
F & W & W & W & F & F & F & F \\
F & W & F & W & F & F & F & F \\
F & F & W & W & F & F & F & F \\
F & F & F & F & F & F & F & F \\
\hline
\end{tabular}

\textbf{Ergebnis}: $A \land (B \lor C)$ und $(A \land B) \lor (A \land C)$ haben in allen Fällen die gleichen Werte, also gilt $A \land (B \lor C) \Leftrightarrow (A \land B) \lor (A \land C)$.

\subsection*{(e)}

\begin{tabular}{|c|c|c|c|c|c|c|}
\hline
$A$ & $B$ & $C$ & $A \rightarrow B$ & $B \rightarrow C$ & $(A \rightarrow B) \land (B \rightarrow C)$ & $A \rightarrow C$ \\
\hline
W & W & W & W & W & W & W \\
W & W & F & W & F & F & F \\
W & F & W & F & W & F & W \\
W & F & F & F & W & F & F \\
F & W & W & W & W & W & W \\
F & W & F & W & F & F & W \\
F & F & W & W & W & W & W \\
F & F & F & W & W & W & W \\
\hline
\end{tabular}

\textbf{Ergebnis}: Wenn $(A \rightarrow B) \land (B \rightarrow C)$ wahr ist, dann ist auch $A \rightarrow C$ wahr. Daher gilt $(A \rightarrow B) \land (B \rightarrow C) \Rightarrow A \rightarrow C$.
 
\section*{Aufgabe 3}

\subsection*{(a)}

\[
A \lor B \equiv (A \mid A) \mid (B \mid B)
\]

\begin{itemize}
    \item $A \mid A = \neg (A \land A) = \neg A$, also ist $(A \mid A)$ eine Negation von $A$.
    \item $B \mid B = \neg (B \land B) = \neg B$, also ist $(B \mid B)$ eine Negation von $B$.
    \item $(A \mid A) \mid (B \mid B) = \neg(\neg A \land \neg B) = A \lor B$ nach den De-Morgan-Gesetzen.
\end{itemize}

\textbf{Ergebnis}: Die Disjunktion $A \lor B$ kann dargestellt werden als $(A \mid A) \mid (B \mid B)$.

\subsection*{(b)}

Betrachten wir den Ausdruck:

\[
((A \mid A) \mid (B \mid B)) \mid ((A \mid A) \mid (B \mid B))
\]

Da wir oben gezeigt haben, dass $(A \mid A) \mid (B \mid B) = A \lor B$, ergibt sich:

\[
((A \lor B) \mid (A \lor B)) = \neg (A \lor B \land A \lor B) = \neg (A \lor B) 
\]

\textbf{Ergebnis}: Die Verknüpfung stellt die NOR-Operation $\neg (A \lor B)$ dar.

\subsection*{(c)}

\begin{enumerate}
  \item $\neg A = A \mid A$
  \item $A \land B = (A \mid B) \mid (A \mid B)$
  \item $A \rightarrow B = (A \mid A) \mid B$
  \item $A \leftrightarrow B = ((A \mid A) \mid B) \mid ((B \mid B) \mid A)$
\end{enumerate}

\section*{Aufgabe 4}

\subsection*{Wahrheitstabelle für die fünf Interpretationen}

\begin{tabular}{|c|c|c|c|c|c|c|c|}
\hline
$A$ & $B$ & $C$ & $A \rightarrow (B \rightarrow C)$ & $(A \rightarrow B) \rightarrow C$ & $(A \land B) \rightarrow C$ & $A \rightarrow (B \land C)$ & $(A \rightarrow B) \land (B \rightarrow C)$ \\
\hline
W & W & W & W & W & W & W & W \\
W & W & F & F & F & F & F & F \\
W & F & W & W & W & W & F & W \\
W & F & F & F & W & F & F & W \\
F & W & W & W & W & W & W & W \\
F & W & F & W & W & W & W & W \\
F & F & W & W & W & W & W & W \\
F & F & F & W & W & W & W & W \\
\hline
\end{tabular}

\subsection*{Analyse der Unterschiede und Fazit}

\begin{itemize}
    \item \textbf{$A \rightarrow (B \rightarrow C)$}: Diese Interpretation entspricht einer Implikation, in der $B \rightarrow C$ der Folgerungsteil ist. Dies ergibt eine wahrheitswertabhängige Aussagekette, die dann wahr ist, wenn $A$ falsch ist oder die Implikation $B \rightarrow C$ wahr ist.
    
    \item \textbf{$(A \rightarrow B) \rightarrow C$}: Hier wird $A \rightarrow B$ als Prämisse genommen. Die gesamte Aussage ist nur dann falsch, wenn $C$ falsch ist und $A \rightarrow B$ wahr ist (d.h., $A$ wahr und $B$ falsch).
    
    \item \textbf{$(A \land B) \rightarrow C$}: Diese Interpretation setzt voraus, dass sowohl $A$ als auch $B$ wahr sein müssen, um $C$ zu implizieren. Dies ist eine stärkere Bedingung, da beide Prämissen erfüllt sein müssen, damit $C$ folgt.
    
    \item \textbf{$A \rightarrow (B \land C)$}: Hier muss $A$ wahr sein, um zu garantieren, dass $B$ und $C$ gleichzeitig wahr sind. Das ist eine andere Bedingung als die einfachen Implikationen.
    
    \item \textbf{$(A \rightarrow B) \land (B \rightarrow C)$}: Diese Interpretation bedeutet, dass sowohl $A$ $B$ impliziert als auch $B$ $C$ impliziert. Beide Bedingungen müssen separat wahr sein, was eine transitive Kette beschreibt.
\end{itemize}

\textbf{Fazit}: Die fünf Interpretationen ergeben je nach Wert von $A$, $B$ und $C$ unterschiedliche Wahrheitsergebnisse, insbesondere bei Kombinationen, bei denen die Implikation nicht eindeutig ist. Es ist daher sinnvoll, die Darstellung $A \rightarrow B \rightarrow C$ zu vermeiden und stattdessen die gewünschte Interpretation explizit auszuschreiben.


\end{document}
%%% Local Variables:
%%% mode: latex
%%% TeX-master: t
%%% End:
