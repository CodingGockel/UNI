% ----------------------- TODO ---------------------------
% Diese Daten müssen pro Blatt angepasst werden:
\newcommand{\NUMBER}{4}
\newcommand{\EXERCISES}{5}
% Diese Daten müssen einmalig pro Vorlesung angepasst werden:
\newcommand{\COURSE}{Diskrete Strukturen I}
\newcommand{\TUTOR}{Jörg Bader}
\newcommand{\STUDENTA}{Moritz Hahn}
\newcommand{\STUDENTB}{}
\newcommand{\STUDENTC}{}
\newcommand{\DEADLINE}{09.11.2024}
% ----------------------- TODO ---------------------------

\documentclass[a4paper]{scrartcl}

\usepackage[utf8]{inputenc}
\usepackage[ngerman]{babel}
\usepackage{amsmath}
\usepackage{amssymb}
\usepackage{fancyhdr}
\usepackage{color}
\usepackage{graphicx}
\usepackage{lastpage}
\usepackage{listings}
\usepackage{tikz}
\usepackage{pdflscape}
\usepackage{subfigure}
\usepackage{float}
\usepackage{polynom}
\usepackage{hyperref}
\usepackage{tabularx}
\usepackage{forloop}
\usepackage{geometry}
\usepackage{listings}
\usepackage{fancybox}
\usepackage{tikz}
\usepackage{amsmath}
\usepackage{amsfonts}
\usepackage{amssymb}
\usepackage{float}
\restylefloat{table}
%Größe der Ränder setzen
\geometry{a4paper,left=3cm, right=3cm, top=3cm, bottom=3cm}

%Kopf- und Fußzeile
\pagestyle {fancy}
\fancyhead[L]{Tutor: \TUTOR}
\fancyhead[C]{\COURSE}
\fancyhead[R]{\today}

\fancyfoot[L]{}
\fancyfoot[C]{} 
\fancyfoot[R]{Seite \thepage /\pageref*{LastPage}}

%Formatierung der Überschrift, hier nichts ändern
\def\header#1#2{
  \begin{center}
    {\Large Übungsblatt #1}\\
    {(Abgabetermin #2)}
  \end{center}
}

%Definition der Punktetabelle, hier nichts ändern
\newcounter{punktelistectr}
\newcounter{punkte}
\newcommand{\punkteliste}[2]{%
  \setcounter{punkte}{#2}%
  \addtocounter{punkte}{-#1}%
  \stepcounter{punkte}%<-- also punkte = m-n+1 = Anzahl Spalten[1]
  \begin{center}%
  \begin{tabularx}{\linewidth}[]{@{}*{\thepunkte}{>{\centering\arraybackslash} X|}@{}>{\centering\arraybackslash}X}
      \forloop{punktelistectr}{#1}{\value{punktelistectr} < #2 } %
      {%
        \thepunktelistectr &
      }
      #2 &  $\Sigma$ \\
      \hline
      \forloop{punktelistectr}{#1}{\value{punktelistectr} < #2 } %
      {%
        &
      } &\\
      \forloop{punktelistectr}{#1}{\value{punktelistectr} < #2 } %
      {%
        &
      } &\\
    \end{tabularx}
  \end{center}
}

\begin{document}

\begin{tabularx}{\linewidth}{m{0.2 \linewidth}X}
  \begin{minipage}{\linewidth}
    \STUDENTA\\
    \STUDENTB\\
    \STUDENTC
  \end{minipage} & \begin{minipage}{\linewidth}
    \punkteliste{1}{\EXERCISES}
  \end{minipage}\\
\end{tabularx}

\header{Nr. \NUMBER}{\DEADLINE}

\section*{Aufgabe 1}

\subsection*{(a)}
\textbf{Lösung:} Die Summe der ersten \( n \) geraden Zahlen \( 2, 4, 6, \ldots, 2n \) kann als das Doppelte der Summe der ersten \( n \) natürlichen Zahlen geschrieben werden:
\[
\sum_{i=1}^n 2i = 2 \sum_{i=1}^n i.
\]
Nach der Gaußschen Summenformel für die Summe der ersten \( n \) natürlichen Zahlen gilt:
\[
\sum_{i=1}^n i = \frac{n(n+1)}{2}.
\]
Einsetzen ergibt:
\[
\sum_{i=1}^n 2i = 2 \cdot \frac{n(n+1)}{2} = n^2 + n.
\]
Damit ist die Aussage bewiesen.

\subsection*{(b)}

\textbf{Lösung:} Die Summe der ersten \( n \) ungeraden Zahlen ist \( 1, 3, 5, \ldots, (2n - 1) \). Es lässt sich durch direktes Aufschreiben und Summieren der Werte feststellen, dass:
\[
\sum_{i=1}^n (2i - 1) = n^2.
\]
Dies kann durch die Beobachtung bestätigt werden, dass jede zusätzliche ungerade Zahl eine vollständige Quadratzahl bildet. Alternativ kann man auch durch Induktion zeigen, dass die Summe der ersten \( n \) ungeraden Zahlen gleich \( n^2 \) ist.

\section*{Aufgabe 2}

\subsection*{(a)}
Für \( n \in \mathbb{N} \) gilt:
\[
\sum_{i=1}^n i^2 = \frac{1}{6} n (n + 1) (2n + 1).
\]
\textbf{Beweis durch vollständige Induktion:}

\textbf{Induktionsanfang:} Für \( n = 1 \):
\[
\sum_{i=1}^1 i^2 = 1 = \frac{1}{6} \cdot 1 \cdot 2 \cdot 3.
\]

\textbf{Induktionsvoraussetzung:} Angenommen, die Aussage gilt für ein \( n \in \mathbb{N} \), also
\[
\sum_{i=1}^n i^2 = \frac{1}{6} n (n + 1) (2n + 1).
\]

\textbf{Induktionsschritt:} Zeigen, dass die Aussage für \( n+1 \) gilt:
\[
\sum_{i=1}^{n+1} i^2 = \frac{1}{6} (n+1)(n+2)(2n+3).
\]
Erweitern der Summe ergibt:
\[
\sum_{i=1}^{n+1} i^2 = \sum_{i=1}^n i^2 + (n+1)^2.
\]
Einsetzen der Induktionsvoraussetzung liefert:
\[
\sum_{i=1}^{n+1} i^2 = \frac{1}{6} n (n + 1) (2n + 1) + (n+1)^2.
\]
Nun bringen wir die Terme auf einen gemeinsamen Nenner und vereinfachen:
\[
\sum_{i=1}^{n+1} i^2 = \frac{1}{6} (n+1)(n+2)(2n+3),
\]
was die Behauptung beweist.

\subsection*{(b)}
\textbf{Beweis durch vollständige Induktion:}

\textbf{Induktionsanfang:} Für \( n = 1 \):
\[
\sum_{k=1}^1 \frac{k}{2^k} = \frac{1}{2} = 2 - \frac{3}{2}.
\]

\textbf{Induktionsvoraussetzung:} Angenommen, die Aussage gilt für ein \( n \in \mathbb{N} \), also
\[
\sum_{k=1}^n \frac{k}{2^k} = 2 - \frac{n + 2}{2^n}.
\]

\textbf{Induktionsschritt:} Zeigen, dass dann auch
\[
\sum_{k=1}^{n+1} \frac{k}{2^k} = 2 - \frac{n+3}{2^{n+1}}.
\]
Durch Hinzufügen des nächsten Terms ergibt sich:
\[
\sum_{k=1}^{n+1} \frac{k}{2^k} = \left(2 - \frac{n+2}{2^n}\right) + \frac{n+1}{2^{n+1}}.
\]
Durch Umformen und Zusammenfassen der Terme erhalten wir:
\[
\sum_{k=1}^{n+1} \frac{k}{2^k} = 2 - \frac{n+3}{2^{n+1}},
\]
womit die Induktionsbehauptung bewiesen ist.

\subsection*{(d)}
Für \( x \geq -1 \) und \( n \in \mathbb{N} \) gilt:
\[
(1 + x)^n \geq 1 + nx.
\]
\textbf{Beweis durch vollständige Induktion:}

\textbf{Induktionsanfang:} Für \( n = 1 \) ist die Aussage offensichtlich wahr:
\[
(1 + x)^1 = 1 + x.
\]

\textbf{Induktionsvoraussetzung:} Angenommen, die Aussage gilt für ein \( n \), also \( (1 + x)^n \geq 1 + nx \).

\textbf{Induktionsschritt:} Zeige die Aussage für \( n+1 \):
\[
(1 + x)^{n+1} = (1 + x)^n (1 + x) \geq (1 + nx)(1 + x).
\]
Durch Ausmultiplizieren und Anwendung der Induktionsvoraussetzung zeigt sich, dass \( (1 + x)^{n+1} \geq 1 + (n+1)x \).

\subsection*{(e)}
\textbf{Induktionsanfang:} Für \( n = 1 \) prüfen wir die Aussage:
\[
(1 + x)^1 = 1 + x.
\]
Da \( 1 + x = 1 + (2 \cdot 1 - 1) \cdot x \), ist die Aussage für \( n = 1 \) erfüllt.

\textbf{Induktionsvoraussetzung:} Angenommen, die Aussage gilt für ein \( n \in \mathbb{N} \), d.h.
\[
(1 + x)^n \leq 1 + (2n - 1) \cdot x.
\]

\textbf{Induktionsschritt:} Zeigen, dass die Aussage dann auch für \( n+1 \) gilt, also:
\[
(1 + x)^{n+1} \leq 1 + (2(n+1) - 1) \cdot x = 1 + (2n + 1) \cdot x.
\]
Wir schreiben \( (1 + x)^{n+1} \) als
\[
(1 + x)^{n+1} = (1 + x)^n \cdot (1 + x).
\]
Anwendung der Induktionsvoraussetzung liefert:
\[
(1 + x)^{n+1} \leq \left( 1 + (2n - 1) \cdot x \right) \cdot (1 + x).
\]
Nun multiplizieren wir die rechte Seite aus:
\[
(1 + (2n - 1) \cdot x) \cdot (1 + x) = 1 + x + (2n - 1) \cdot x + (2n - 1) \cdot x^2.
\]
Da \( x \in [0, 1] \), folgt \( x^2 \leq x \). Damit können wir abschätzen:
\[
1 + x + (2n - 1) \cdot x + (2n - 1) \cdot x^2 \leq 1 + x + (2n - 1) \cdot x + (2n - 1) \cdot x = 1 + (2n + 1) \cdot x.
\]
Damit ist die Aussage auch für \( n+1 \) wahr, und der Beweis ist abgeschlossen.

\section*{Aufgabe 3}

Beachten wir, dass sich \( 1 + \frac{1}{k} = \frac{k+1}{k} \) schreiben lässt. Daher ergibt sich:
\[
f(n) = \prod_{k=1}^n \frac{k+1}{k}.
\]
Durch Ausmultiplizieren des Produkts erkennen wir, dass viele Terme im Zähler und Nenner wegfallen (Teleskopprodukt):
\[
f(n) = \frac{2}{1} \cdot \frac{3}{2} \cdot \frac{4}{3} \cdots \frac{n+1}{n} = \frac{n+1}{1} = n+1.
\]
Somit erhalten wir die explizite Formel:
\[
f(n) = n + 1.
\]

\subsection*{Beweis durch vollständige Induktion}
\textbf{Induktionsanfang:} Für \( n = 1 \) ergibt sich:
\[
f(1) = 1 + 1 = 2,
\]
was mit der Formel \( f(n) = n + 1 \) übereinstimmt.

\textbf{Induktionsvoraussetzung:} Angenommen, die Formel gilt für ein \( n \in \mathbb{N} \), also \( f(n) = n + 1 \).

\textbf{Induktionsschritt:} Zeigen, dass dann auch \( f(n+1) = (n+1) + 1 = n + 2 \) gilt.
\[
f(n+1) = f(n) \cdot \left(1 + \frac{1}{n+1}\right).
\]
Nach Induktionsvoraussetzung ist \( f(n) = n + 1 \), sodass
\[
f(n+1) = (n+1) \cdot \frac{n+2}{n+1} = n + 2.
\]
Damit ist die Aussage bewiesen.

\section*{Aufgabe 4}

\subsection*{(a)}
\begin{itemize}
    \item Für \( n = 1 \) ergibt sich eine Gerade, die die Ebene in 2 Gebiete teilt, also \( e_1 = 2 \).
    \item Für \( n = 2 \) schneiden zwei Geraden die Ebene in 4 Gebiete, also \( e_2 = 4 \).
    \item Für \( n = 3 \) teilen drei Geraden die Ebene in 7 Gebiete, also \( e_3 = 7 \).
    \item Für \( n = 4 \) teilen vier Geraden die Ebene in 11 Gebiete, also \( e_4 = 11 \).
\end{itemize}

\subsection*{(b) Rekursive Berechnungsvorschrift für \( e_{n+1} \)}
Betrachten wir den Übergang von \( n \) zu \( n+1 \) Geraden. Die \( n+1 \)-te Gerade schneidet jede der \( n \) bestehenden Geraden genau einmal und fügt somit \( n+1 \) neue Gebiete hinzu. Daher gilt:
\[
e_{n+1} = e_n + (n + 1).
\]

\subsection*{(c) Beweis der expliziten Formel}
Wir sollen zeigen, dass
\[
e_n = \frac{1}{2}n^2 + \frac{1}{2}n + 1.
\]
\textbf{Beweis durch vollständige Induktion:}

\textbf{Induktionsanfang:} Für \( n = 1 \) ist
\[
e_1 = 2 = \frac{1}{2} \cdot 1^2 + \frac{1}{2} \cdot 1 + 1 = 2,
\]
was stimmt.

\textbf{Induktionsvoraussetzung:} Angenommen, die Formel gilt für ein \( n \in \mathbb{N} \), also
\[
e_n = \frac{1}{2}n^2 + \frac{1}{2}n + 1.
\]

\textbf{Induktionsschritt:} Zeigen, dass dann auch \( e_{n+1} = \frac{1}{2}(n+1)^2 + \frac{1}{2}(n+1) + 1 \) gilt.
Nach der rekursiven Beziehung aus Teil (b) gilt:
\[
e_{n+1} = e_n + (n+1).
\]
Durch Einsetzen der Induktionsvoraussetzung ergibt sich:
\[
e_{n+1} = \left(\frac{1}{2}n^2 + \frac{1}{2}n + 1\right) + (n+1).
\]
Durch Vereinfachung erhalten wir:
\[
e_{n+1} = \frac{1}{2}n^2 + \frac{1}{2}n + 1 + n + 1 = \frac{1}{2}n^2 + \frac{3}{2}n + 2.
\]
Nun schreiben wir \( \frac{1}{2}n^2 + \frac{3}{2}n + 2 \) um:
\[
e_{n+1} = \frac{1}{2}(n+1)^2 + \frac{1}{2}(n+1) + 1,
\]
was die Behauptung zeigt.

\section*{Aufgabe 5: Fehlerhafter Beweis für Schnittpunkt aller Geraden}
Der Fehler liegt darin, dass bei \( m + 1 \) Geraden nicht garantiert ist, dass sich alle in einem gemeinsamen Punkt schneiden, da allgemeine Lage paarweise nicht-parallele Geraden nicht notwendigerweise einen gemeinsamen Schnittpunkt haben.

\end{document}
%%% Local Variables:
%%% mode: latex
%%% TeX-master: t
%%% End:
