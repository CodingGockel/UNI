% ----------------------- TODO ---------------------------
% Diese Daten müssen pro Blatt angepasst werden:
\newcommand{\NUMBER}{5}
\newcommand{\EXERCISES}{3}
% Diese Daten müssen einmalig pro Vorlesung angepasst werden:
\newcommand{\COURSE}{Diskrete Strukturen I}
\newcommand{\TUTOR}{Jörg Bader}
\newcommand{\STUDENTA}{Moritz Hahn}
\newcommand{\STUDENTB}{}
\newcommand{\STUDENTC}{}
\newcommand{\DEADLINE}{16.11.2024}
% ----------------------- TODO ---------------------------

\documentclass[a4paper]{scrartcl}

\usepackage[utf8]{inputenc}
\usepackage[ngerman]{babel}
\usepackage{amsmath}
\usepackage{amssymb}
\usepackage{fancyhdr}
\usepackage{color}
\usepackage{graphicx}
\usepackage{lastpage}
\usepackage{listings}
\usepackage{tikz}
\usepackage{pdflscape}
\usepackage{subfigure}
\usepackage{float}
\usepackage{polynom}
\usepackage{hyperref}
\usepackage{tabularx}
\usepackage{forloop}
\usepackage{geometry}
\usepackage{listings}
\usepackage{fancybox}
\usepackage{tikz}
\usepackage{amsmath}
\usepackage{amsfonts}
\usepackage{amssymb}
\usepackage{float}
\restylefloat{table}
%Größe der Ränder setzen
\geometry{a4paper,left=3cm, right=3cm, top=3cm, bottom=3cm}

%Kopf- und Fußzeile
\pagestyle {fancy}
\fancyhead[L]{Tutor: \TUTOR}
\fancyhead[C]{\COURSE}
\fancyhead[R]{\today}

\fancyfoot[L]{}
\fancyfoot[C]{} 
\fancyfoot[R]{Seite \thepage /\pageref*{LastPage}}

%Formatierung der Überschrift, hier nichts ändern
\def\header#1#2{
  \begin{center}
    {\Large Übungsblatt #1}\\
    {(Abgabetermin #2)}
  \end{center}
}

%Definition der Punktetabelle, hier nichts ändern
\newcounter{punktelistectr}
\newcounter{punkte}
\newcommand{\punkteliste}[2]{%
  \setcounter{punkte}{#2}%
  \addtocounter{punkte}{-#1}%
  \stepcounter{punkte}%<-- also punkte = m-n+1 = Anzahl Spalten[1]
  \begin{center}%
  \begin{tabularx}{\linewidth}[]{@{}*{\thepunkte}{>{\centering\arraybackslash} X|}@{}>{\centering\arraybackslash}X}
      \forloop{punktelistectr}{#1}{\value{punktelistectr} < #2 } %
      {%
        \thepunktelistectr &
      }
      #2 &  $\Sigma$ \\
      \hline
      \forloop{punktelistectr}{#1}{\value{punktelistectr} < #2 } %
      {%
        &
      } &\\
      \forloop{punktelistectr}{#1}{\value{punktelistectr} < #2 } %
      {%
        &
      } &\\
    \end{tabularx}
  \end{center}
}

\begin{document}

\begin{tabularx}{\linewidth}{m{0.2 \linewidth}X}
  \begin{minipage}{\linewidth}
    \STUDENTA\\
    \STUDENTB\\
    \STUDENTC
  \end{minipage} & \begin{minipage}{\linewidth}
    \punkteliste{1}{\EXERCISES}
  \end{minipage}\\
\end{tabularx}

\header{Nr. \NUMBER}{\DEADLINE}

\section*{Aufgabe 1:}

\subsection*{(a) Drei Mengen \( A, B, C \)}
Die möglichen Regionen im Venn-Diagramm mit drei Mengen lassen sich wie folgt notieren:
\begin{enumerate}
    \item Nur in \( A \): \( A \setminus (B \cup C) \)
    \item Nur in \( B \): \( B \setminus (A \cup C) \)
    \item Nur in \( C \): \( C \setminus (A \cup B) \)
    \item In \( A \cap B \), aber nicht in \( C \): \( (A \cap B) \setminus C \)
    \item In \( A \cap C \), aber nicht in \( B \): \( (A \cap C) \setminus B \)
    \item In \( B \cap C \), aber nicht in \( A \): \( (B \cap C) \setminus A \)
    \item In \( A \cap B \cap C \): \( A \cap B \cap C \)
    \item Außerhalb aller Mengen: \( \overline{A \cup B \cup C} \)
\end{enumerate}

\subsection*{(b) Vier Mengen \( A, B, C, D \)}
Die fehlenden Regionen im Venn-Diagramm mit vier Mengen lauten:
\begin{enumerate}
    \item Nur in \( A \) und \( D \): \( (A \cap D) \setminus (B \cup C) \)
    \item Nur in \( B \) und \( D \): \( (B \cap D) \setminus (A \cup C) \)
    \item Nur in \( C \) und \( D \): \( (C \cap D) \setminus (A \cup B) \)
    \item In allen vier Mengen: \( A \cap B \cap C \cap D \)
\end{enumerate}

\section*{Aufgabe 2:}

\subsection*{(a) \( A \cup B = A \cap B \)}
\textbf{Behauptung:} Falsch. \\
\textbf{Gegenbeispiel:} Sei \( A = \{1\} \), \( B = \{2\} \). Dann ist:
\[
A \cup B = \{1, 2\}, \quad A \cap B = \emptyset.
\]
Also gilt \( A \cup B \neq A \cap B \).

\subsection*{(b) \( A \cap (B \cup C) = (A \cap B) \cup C \)}
\textbf{Behauptung:} Falsch. \\
\textbf{Gegenbeispiel:} Sei \( A = \{1\} \), \( B = \{2\} \), \( C = \{3\} \). Dann:
\[
A \cap (B \cup C) = \emptyset, \quad (A \cap B) \cup C = \{3\}.
\]
Daher \( A \cap (B \cup C) \neq (A \cap B) \cup C \).

\subsection*{(c) \( A \cup B = A \cap B \iff A = B \)}
\textbf{Behauptung:} Wahr. \\
\textbf{Beweis:} \( A \cup B = A \cap B \) bedeutet:
\[
x \in A \cup B \Leftrightarrow x \in A \cap B.
\]
Das impliziert \( x \in A \Leftrightarrow x \in B \), also \( A = B \).

\subsection*{(d) \( A \cup (B \cap C) = (A \cup B) \cap (A \cup C) \)}
\textbf{Behauptung:} Wahr. \\
\textbf{Beweis:} Zeige Äquivalenz der beiden Seiten:
\begin{align*}
\text{Links: } & x \in A \cup (B \cap C) \Leftrightarrow x \in A \vee (x \in B \wedge x \in C), \\
\text{Rechts: } & x \in (A \cup B) \cap (A \cup C) \Leftrightarrow (x \in A \vee x \in B) \wedge (x \in A \vee x \in C).
\end{align*}
Beide Aussagen sind identisch.

\subsection*{(e) \( (A \cup B) \cap (B \cup C) \cap (C \cup A) = (A \cap B) \cup (B \cap C) \cup (C \cap A) \)}
\textbf{Behauptung:} Wahr. \\
\textbf{Beweis:} Durch Anwendung des Distributivgesetzes zeigt sich, dass beide Seiten äquivalent sind.

\subsection*{(f) \( P(A) \cap P(B) = P(A \cap B) \iff A \subseteq B \lor B \subseteq A \lor A \cap B = \emptyset \)}
\textbf{Behauptung:} Wahr. \\
\textbf{Beweis:} Die Potenzmengen \( P(A) \cap P(B) \) und \( P(A \cap B) \) stimmen nur überein, wenn eine der genannten Bedingungen erfüllt ist.

\section*{Aufgabe 3:}

Zeige, dass die Aussagen äquivalent sind:
\begin{enumerate}
    \item \( M \subseteq N \),
    \item \( M \cup N = N \),
    \item \( M \cap N = M \).
\end{enumerate}

\textbf{Beweis:}
\begin{itemize}
    \item \( M \subseteq N \Rightarrow M \cup N = N \): \\
    Elemente von \( M \) sind in \( N \), also ergänzt \( M \) nichts.
    \item \( M \cup N = N \Rightarrow M \cap N = M \): \\
    Elemente von \( M \) sind in \( N \), daher ist \( M \cap N = M \).
    \item \( M \cap N = M \Rightarrow M \subseteq N \): \\
    Jedes Element von \( M \) liegt in \( N \), also \( M \subseteq N \).
\end{itemize}
\end{document}
%%% Local Variables:
%%% mode: latex
%%% TeX-master: t
%%% End:
