% ----------------------- TODO ---------------------------
% Diese Daten müssen pro Blatt angepasst werden:
\newcommand{\NUMBER}{6}
\newcommand{\EXERCISES}{5}
% Diese Daten müssen einmalig pro Vorlesung angepasst werden:
\newcommand{\COURSE}{Diskrete Strukturen I}
\newcommand{\TUTOR}{Jörg Bader}
\newcommand{\STUDENTA}{Moritz Hahn}
\newcommand{\STUDENTB}{}
\newcommand{\STUDENTC}{}
\newcommand{\DEADLINE}{23.11.2024}
% ----------------------- TODO ---------------------------

\documentclass[a4paper]{scrartcl}

\usepackage[utf8]{inputenc}
\usepackage[ngerman]{babel}
\usepackage{amsmath}
\usepackage{amssymb}
\usepackage{fancyhdr}
\usepackage{color}
\usepackage{graphicx}
\usepackage{lastpage}
\usepackage{listings}
\usepackage{tikz}
\usepackage{pdflscape}
\usepackage{subfigure}
\usepackage{float}
\usepackage{polynom}
\usepackage{hyperref}
\usepackage{tabularx}
\usepackage{forloop}
\usepackage{geometry}
\usepackage{listings}
\usepackage{fancybox}
\usepackage{tikz}
\usepackage{amsmath}
\usepackage{amsfonts}
\usepackage{amssymb}
\usepackage{float}
\restylefloat{table}
%Größe der Ränder setzen
\geometry{a4paper,left=3cm, right=3cm, top=3cm, bottom=3cm}

%Kopf- und Fußzeile
\pagestyle {fancy}
\fancyhead[L]{Tutor: \TUTOR}
\fancyhead[C]{\COURSE}
\fancyhead[R]{\today}

\fancyfoot[L]{}
\fancyfoot[C]{} 
\fancyfoot[R]{Seite \thepage /\pageref*{LastPage}}

%Formatierung der Überschrift, hier nichts ändern
\def\header#1#2{
  \begin{center}
    {\Large Übungsblatt #1}\\
    {(Abgabetermin #2)}
  \end{center}
}

%Definition der Punktetabelle, hier nichts ändern
\newcounter{punktelistectr}
\newcounter{punkte}
\newcommand{\punkteliste}[2]{%
  \setcounter{punkte}{#2}%
  \addtocounter{punkte}{-#1}%
  \stepcounter{punkte}%<-- also punkte = m-n+1 = Anzahl Spalten[1]
  \begin{center}%
  \begin{tabularx}{\linewidth}[]{@{}*{\thepunkte}{>{\centering\arraybackslash} X|}@{}>{\centering\arraybackslash}X}
      \forloop{punktelistectr}{#1}{\value{punktelistectr} < #2 } %
      {%
        \thepunktelistectr &
      }
      #2 &  $\Sigma$ \\
      \hline
      \forloop{punktelistectr}{#1}{\value{punktelistectr} < #2 } %
      {%
        &
      } &\\
      \forloop{punktelistectr}{#1}{\value{punktelistectr} < #2 } %
      {%
        &
      } &\\
    \end{tabularx}
  \end{center}
}

\begin{document}

\begin{tabularx}{\linewidth}{m{0.2 \linewidth}X}
  \begin{minipage}{\linewidth}
    \STUDENTA\\
    \STUDENTB\\
    \STUDENTC
  \end{minipage} & \begin{minipage}{\linewidth}
    \punkteliste{1}{\EXERCISES}
  \end{minipage}\\
\end{tabularx}

\header{Nr. \NUMBER}{\DEADLINE}

\section*{Aufgabe 1:}

\subsection*{(a)}
Die Elemente der Menge \( X \times Y \) sind:
\[
X \times Y = \{(a, 1), (a, 2), (a, 3), (a, 4), (a, 5), (b, 1), (b, 2), \dots, (e, 5)\}.
\]
Insgesamt gibt es \( |X| \cdot |Y| = 5 \cdot 5 = 25 \) Elemente.

\subsection*{(b)}
Die Potenzmenge \( P(X \times Y) \) hat \( 2^{|X \times Y|} \) Elemente. Da \( |X \times Y| = 25 \), ergibt sich:
\[
|P(X \times Y)| = 2^{25}.
\]

\subsection*{(c)}
\begin{itemize}
    \item \textbf{Funktionen:} Für \( f : X \to Y \) gibt es \( |Y|^{|X|} = 5^5 \) Möglichkeiten:
    \[
    5^5 = 3125.
    \]

    \item \textbf{Injektionen:} Eine injektive Funktion ist eine Anordnung von \( |X| = 5 \) Elementen auf \( |Y| = 5 \) Plätzen:
    \[
    P(5, 5) = 5! = 120.
    \]

    \item \textbf{Surjektionen:} Für surjektive Funktionen verwenden wir die Inklusions-Exklusionsformel:
    \[
    S(5, 5) = \sum_{k=0}^5 (-1)^k \binom{5}{k} (5-k)^5 = 120.
    \]

    \item \textbf{Bijektionen:} Eine Funktion ist bijektiv, wenn sie sowohl injektiv als auch surjektiv ist. Es gibt:
    \[
    5! = 120 \text{ Bijektionen.}
    \]
\end{itemize}

\subsection*{(d)}
Solche Bijektionen entsprechen involutorischen Permutationen, die nur aus Zyklen der Länge 1 oder 2 bestehen. Für \( |X| = 5 \) gilt:
\[
\sum_{k=0}^{\lfloor n/2 \rfloor} \frac{n!}{k! \cdot (n-2k)! \cdot 2^k}.
\]
Für \( n = 5 \) ergibt dies:
\[
15 \text{ involutorische Bijektionen.}
\]

\section*{Aufgabe 2:}
Eine Funktion \( f : X \to Y \) entspricht der Relation:
\[
\{(x, f(x)) \mid x \in X\}.
\]
Diese Relation hat folgende Eigenschaften:
\begin{itemize}
    \item Sie ist \textbf{linkstotal}, da jedes \( x \in X \) einem Wert in \( Y \) zugeordnet wird.
    \item Sie ist \textbf{rechtseindeutig}, da jedes \( x \in X \) höchstens einem Wert in \( Y \) zugeordnet wird.
\end{itemize}

\subsection*{Injektive Funktionen}
Eine Relation ist injektiv, wenn keine zwei Tupel denselben zweiten Wert besitzen.

\subsection*{Surjektive Funktionen}
Eine Relation ist surjektiv, wenn jedes \( y \in Y \) in mindestens einem Tupel auftritt.

\section*{Aufgabe 3:}
Gegeben ist die Relation:
\[
R = \{(1, 1), (2, 1), (2, 3), (3, 2)\}.
\]

\subsection*{(a)}
Eine Äquivalenzrelation ist reflexiv, symmetrisch und transitiv:
\begin{itemize}
    \item \textbf{Reflexivität:} Ergänzen der Paare \((x, x)\) für \( x \in \{4, 5\} \): 
    \[
    R_1 = \{(4, 4), (5, 5)\}.
    \]
    \item \textbf{Symmetrie:} Ergänzen der Paare \((1, 2), (3, 2), (2, 1)\).
    \item \textbf{Transitivität:} Ergänzen der Paare \((2, 2), (3, 1)\).
\end{itemize}
Die finale Relation ist:
\[
R \cup R_1 = \{(1, 1), (2, 1), (1, 2), (2, 2), (2, 3), (3, 2), (3, 1), (4, 4), (5, 5)\}.
\]

\subsection*{(b)}
Eine Halbordnungsrelation ist reflexiv, antisymmetrisch und transitiv:
\begin{itemize}
    \item \textbf{Reflexivität:} Ergänzen der Paare \((4, 4), (5, 5)\):
    \[
    R_2 = \{(4, 4), (5, 5)\}.
    \]
    \item \textbf{Antisymmetrie:} Es sind keine widersprüchlichen Paare vorhanden.
    \item \textbf{Transitivität:} Ergänzen des Paares \((3, 1)\).
\end{itemize}
Die finale Relation ist:
\[
R \cup R_2 = \{(1, 1), (2, 1), (2, 2), (2, 3), (3, 3), (3, 1), (4, 4), (5, 5)\}.
\]

\subsection*{(c)}
\begin{itemize}
    \item \textbf{Transitivität:} Ergänzen der Paare \((3, 1), (2, 2), (3, 3)\).
    \item \textbf{Nicht asymmetrisch:} Ergänzen des Paares \((1, 2)\).
\end{itemize}
Die finale Relation ist:
\[
R \cup R_3 = \{(1, 1), (2, 1), (1, 2), (2, 3), (3, 2), (3, 1), (3, 3)\}.
\]

\subsection*{(d)}
\begin{itemize}
    \item \textbf{Nicht reflexiv:} Entfernen des Paares \((1, 1)\).
    \item \textbf{Nicht transitiv:} Keine Ergänzung transitiver Paare wie \((2, 2)\) oder \((3, 1)\).
\end{itemize}
Die finale Relation ist:
\[
R \cup R_4 = \{(2, 1), (2, 3), (3, 2)\}.
\]

\section*{Aufgabe 4:}

\subsection*{(a)}
\begin{itemize}
    \item \textbf{Symmetrie:} Wenn \((a, b) \in R\), dann auch \((b, a) \in R\).
    \item \textbf{Antisymmetrie:} Wenn \((a, b) \in R\) und \((b, a) \in R\), dann gilt \( a = b \).
\end{itemize}
Wegen Antisymmetrie bleibt nur \( a = b \). Reflexivität folgt aus \( a = b \). Somit gilt für \((a, b), (b, c) \in R\), dass auch \((a, c) \in R\).  
\textbf{Folgerung:} \( R \) ist transitiv.

\subsection*{(b)}
Zu zeigen: \((R \circ S) \circ T = R \circ (S \circ T)\).  
Sei \( x \in (R \circ S) \circ T \). Dann existieren \( y, z \) mit:
\[
(x, y) \in R, \ (y, z) \in S, \ (z, w) \in T.
\]
Analog gilt für \( R \circ (S \circ T) \). Die Assoziativität folgt aus der Definition des Relationenprodukts.

\subsection*{(c)}
Sei \( (a, b) \in (R \circ S)^{-1} \). Dann gilt:
\[
(b, a) \in R \circ S \Rightarrow \exists c: (b, c) \in R, \ (c, a) \in S.
\]
Das bedeutet:
\[
(a, c) \in S^{-1}, \ (c, b) \in R^{-1}.
\]
Somit ist \( (a, b) \in S^{-1} \circ R^{-1} \).

\section*{Aufgabe 5:}

\subsection*{(a)}
\( F \cap G \) ist nur dann eine Funktion, wenn \( F(x) = G(x) \) für alle \( x \in \text{Dom}(F) \cap \text{Dom}(G) \).  
\textbf{Definitionsmenge:} \( \{x \in X \mid F(x) = G(x)\} \).

\subsection*{(b)}
\( F \cup G \) ist nur dann eine Funktion, wenn \( F(x) = G(x) \) für alle \( x \in \text{Dom}(F) \cap \text{Dom}(G) \).  
\textbf{Definitionsmenge:} \( \text{Dom}(F) \cup \text{Dom}(G) \).

\end{document}
%%% Local Variables:
%%% mode: latex
%%% TeX-master: t
%%% End:
